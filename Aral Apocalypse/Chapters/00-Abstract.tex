\thispagestyle{plain}

\pdfbookmark[1]{Executive Summary}{abstract}
\chapter*{Executive Summary}
The Aral Sea Crisis represents one of the most significant cases of anthropogenic environmental degradation in recent history.  Once the world’s fourth-largest inland lake, the Aral Sea has declined by over 85\% since the 1960s, primarily due to Soviet-era irrigation policies aimed at expanding cotton production. The diversion of the Amu Darya and Syr Darya rivers led to major ecological disruptions, including lakebed desiccation, biodiversity loss, and regional climate shifts. Local communities continue to face respiratory illnesses, water shortages, economic decline, and displacement (WHO, 2011; UNEP, 2019). While the Kok-Aral Dam has enabled partial recovery in the North Aral Sea, the southern basin remains a stark symbol of irreversible environmental mismanagement. \\
\\The crisis serves as a cautionary tale, especially in the context of climate change-driven water insecurity. Today, similar pressures are mounting in vital water systems globally—from the Great Lakes in North America, facing eutrophication and fluctuating water levels, to the Colorado River Basin, which is drying under the strain of overuse and rising temperatures. \\
\\This report uses the Aral Sea as a lens to explore broader themes of sustainable engineering, water ethics, and international environmental governance. It underscores the urgency for climate-resilient infrastructure, transboundary water cooperation, and community-driven adaptation strategies, particularly as global warming intensifies competition over finite freshwater resources. In the age of climate change, the Aral Sea is not just a regional tragedy—it is a global warning. 

