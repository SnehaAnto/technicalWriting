\chapter[Cotton Production and the Soviet Irrigation Plan]{Cotton Production and the Soviet Irrigation Plan}
\label{cp:cotton-production}

In the 1960s, the Soviet Union embarked on an ambitious plan to increase cotton production in Central Asia, transforming the region’s desert landscapes into highly productive agricultural zones.\autocite{ebsco_aral} Cotton, referred to as 'White Gold', was considered an essential export for the USSR, and the government’s drive to increase yields led to the implementation of large-scale irrigation systems that diverted water from the Amu Darya and Syr Darya rivers, which were the primary sources feeding the Aral Sea \autocite{ejf_whitegold}.
The Soviet authorities constructed an extensive network of canals to facilitate this massive irrigation effort. However, the system was riddled with inefficiencies. Around 60\% of the diverted water was lost due to poorly designed and unlined canals that leaked or evaporated before reaching the fields \autocite{ejf_whitegold}\autocite{grabish_drytears}. Despite these inefficiencies, cotton production in the region surged. By the 1980s, the Soviet Union had transformed Central Asia into one of the world’s leading cotton-producing regions \autocite{conrad_irrigated}.

However, this growth came at a severe environmental cost. The diversion of water to irrigate cotton fields meant that less water flowed into the Aral Sea, causing its water levels to plummet. By the 1990s, the Aral Sea had lost over 90\% of its volume, triggering a cascade of ecological and social consequences \autocite{rasmussen_nasa}. The rising salinity in the remaining water rendered the lake increasingly inhospitable to aquatic life \autocite{plotnikov_fauna}. The pursuit of \textit{`white gold'}, ultimately decimated the Aral Sea, exposing the unsustainable nature of the Soviet irrigation plan.