

\chapter[Global Parallels in Water Mismanagement]{Global Parallels in Water Mismanagement}
\label{cp:global-parallels}
\vspace{.935em}

The Aral Sea crisis is part of a global pattern of water mismanagement, driven by over-extraction, poor planning, and climate change. Several prominent water bodies around the world are facing similar declines (see \autoref{cp:lake-distress}), illustrating the widespread risks of neglecting sustainable water management.

\section{Colorado River (U.S./Mexico)}
The Colorado River, a critical water source for millions, is over-allocated, leading to historically low water levels in reservoirs like Lake Mead and Lake Powell. Its depletion,
worsened by climate change and reduced snowpack, has caused ecological damage and water security risks for the southwestern U.S. and Mexico \autocite{udall2017colorado}.

\section{Lake Chad (Africa)}
Lake Chad has shrunk by over 90\% since the 1960s due to excessive irrigation, population growth, and climate change. These factors have led to severe water scarcity, affecting millions and contributing to socio-economic instability and conflict \autocite{unep2025tale}\autocite{okpara2016lake}.

\section{Lake Urmia (Iran)}
Once the largest saltwater lake in the Middle East, Lake Urmia has drastically shrunk due to dam construction, water diversion for agriculture, and poor water management. The environmental and socio-economic consequences have been devastating for local
communities \autocite{mahoozi2024urmia}.

\section{The Ganges, Nile, and Yangtze Rivers}
    These iconic rivers are under severe stress: 
\begin{itemize}
    \item Ganges: Over-extraction, pollution, and glacial melting threaten water availability for millions in India and Bangladesh \autocite{wwf2025ganges}\autocite{rai2024ganga}.
    \item  Nile: Tensions over the Grand Ethiopian Renaissance Dam and climate change exacerbate water scarcity, threatening food security in northeastern Africa \autocite{mbaku2023dam}\autocite{fayoumi2023impact}. 
    \item Yangtze: Over-extraction, pollution, and the impacts of large-scale dams threaten the river’s ecosystem and millions of livelihoods \autocite{mehra2024yangtze}.
\end{itemize}
     
These global parallels highlight the urgent need for sustainable, integrated water management practices to prevent further environmental and human suffering. 