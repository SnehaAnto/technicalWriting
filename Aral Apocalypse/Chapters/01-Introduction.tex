

\chapter[Introduction]{Introduction}
\label{cp:introduction}

\vspace{.935em}

The Aral Sea crisis exemplifies the dangers posed by the overexploitation and mismanagement of freshwater resources. Human activity, driven by agricultural needs, political goals, and economic pressures, has significantly diminished global water supplies, especially in arid and semi-arid regions. The Aral Sea—technically a lake but historically referred to as a \textit{`sea'} due to its vast size and saline water—is now a desert, having faced one of the most extreme cases of environmental degradation, stemming from poor governance, unsustainable irrigation practices, and the failure to consider the complex interactions between water management and climate variability.\autocite{natgeo_aral} Climate change further exacerbates these challenges by increasing evaporation rates, reducing precipitation, and intensifying water scarcity. As global freshwater systems are increasingly threatened, the lessons from the Aral Sea serve as a stark warning of how unsustainable water management can lead to irreversible ecological and socioeconomic impacts. This report delves into the causes, consequences, and restoration efforts of the Aral Sea disaster, offering insights into how better water management practices could have mitigated this crisis and can inform global water governance today. 

